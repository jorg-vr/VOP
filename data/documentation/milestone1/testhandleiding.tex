\documentclass[12pt,a4paper,twoside]{article}
\usepackage[dutch]{babel}
%laad de pakketten nodig om wiskunde weer te geven :
\usepackage{amsmath,amssymb,amsfonts,textcomp}
%laad de pakketten voor figuren :
\usepackage{multirow}
\usepackage{graphicx}
\usepackage{float,flafter}
\usepackage{hyperref}
\usepackage{inputenc}
%zet de bladspiegel :
\setlength\paperwidth{20.999cm}\setlength\paperheight{29.699cm}\setlength\voffset{-1in}\setlength\hoffset{-1in}\setlength\topmargin{1.499cm}\setlength\headheight{12pt}\setlength\headsep{0cm}\setlength\footskip{1.131cm}\setlength\textheight{25cm}\setlength\oddsidemargin{2.499cm}\setlength\textwidth{15.999cm}

\begin{document}
\noindent
\begin{center}
{\bf {\Huge Testhandleiding}}
\end{center}
\noindent


\section{Lagen}
\subsection{Presentatielaag}
Voor de presentatielaag zijn er momenteel geen geautomatiseerde testen en wordt manueel testen toegepast.


\subsection{Applicatie laag (controller/creator patterns)}
De testen voor spring controllers voeren gesimmuleerde HTTP requests uit. Hierbij testen ze of data kan worden opgevraagd, aangepast en verwijderd. Daarbij wordt ook getest of eventuele wijziggingen behouden blijven.

\subsection{Domein laag}
In de domein laag valt er voor milestone 1 nog niet veel te testen gezien er buiten schrijven, opvragen, updaten en verwijderen van data nog geen andere logica ge{\"i}mplementeerd is. De tests in de domein laag beperken zich momenteel tot unit tests die het correct werken van de data-validatie (zoals nagaan of een chassisnummer het juiste formaat heeft) in de models controleren.

\subsection{Persistence laag}
De persistence laag wordt momenteel getest aan de hand van integratietests op de verschillende methodes van de DAO's. Er wordt momenteel enkel getest of er succesvol data kan toegevoegd/opgevraagd/geupdate/verwijderd worden uit de database en of alle filters voor het opvragen van specifieke data correct werken. Er zijn nog geen tests die de consistentie van de database zelf nagaan.

\section{Uitvoeren van de tests}
%nog aan te vullen want ik weet bijgod niet oe ge met de commandline werkt vo java code uit te voeren...
Voor het testen van de backend wordt gebruik gemaakt van het Junit test framework. Met het volgende commando kan men alle tests uitvoeren. Indien men enkel de tests op de models, DAOs of controllers wenst uit te voeren geeft men dit mee als optie~:\\
\begin{center}
	\texttt{java TestRunner [model, dao, controller]}
\end{center}


\end{document}